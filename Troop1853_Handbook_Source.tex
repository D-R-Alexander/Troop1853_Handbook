\documentclass{ltxguide}
\usepackage{helvet}
\usepackage[margin=1in]{geometry}
\usepackage{url}

\usepackage{titlesec}
\setcounter{secnumdepth}{4}
\renewcommand{\familydefault}{\sfdefault}
\setcounter{tocdepth}{5}

\usepackage[printonlyused]{acronym}

\title{Troop 1853 Handbook}
\date{March 1, 2023}
%\date{\today}
\begin{document}

\maketitle
\pagenumbering{gobble}
\newpage

\section*{Preface}
The Troop 1853 Handbook is designed to describe in detail what Troop 1853 does and how it operates within the guidelines and regulations of the Chartered Organization and the \ac{BSA}. This handbook is approved by the Troop committee and is applicable to both the \ac{BT} and the \ac{GT}.

This handbook is intended to work with other \ac{BSA} and Troop 1853 policies and is found by the links provided within in the document or on the troop’s websites: 

For \ac{BT}: \url{https://springfield1853.mytroop.us or https://www.troop1853.org}

For \ac{GT}: \url{https://troopwebhost.org/Troop1853SPRINGFIELD/ or www.troop1853.com}

This handbook is meant to help guide the actions and activities of Troop 1853. Nothing is intended to add to requirements established by the \ac{BSA}, the \ac{NCAC}, and the \ac{ODD}.

The Troop Committee is responsible for changes and approval of this handbook. Any registered member of Troop 1853 may propose a change to any part of this document, by presenting the change to the Troop Committee at their monthly committee meeting.

\newpage
\section*{Dear Parents, Scouters, and Scouts of Troop 1853,}

This handbook provides information, which will assist you in getting the most out of your Scouting experience – whether you are a parent, Scouter, or Scout. The handbook outlines our policies and describes how we build and implement our program. This is a living document, and it is updated as rules and procedures evolve. It is intended as a supplement to the current \ac{BSA} rules, regulations, and publications. It explains how Troop 1853 operates and is applicable to both the \ac{BT} and the \ac{GT}.
 
The handbook is available on the troop website:
\url{https://springfield1853.mytroop.us//} or \url{https://www.troop1853.org} for the \ac{BT} and \url{https://troopwebhost.org/Troop1853SPRINGFIELD/} or \url{www.troop1853.com} for the \ac{GT}. 

If you have any questions, which are not covered in this handbook, please let me or one of the Scoutmasters know. We have a host of Scouters who can help find the answer.

Remember, none of the Scouter positions in the troop – The Committee, Scoutmaster, \acp{ASM}, or Patrol Advisors -- is gender-specific! We invite both mom and dad to help us!

We will include the resolution of any issues in the next update to the handbook. Please send any recommended changes to the Committee Chair. Our sole purpose is to support our Scouts and provide the best Scouting program possible. We are always interested in additional adult assistance. Please make some time to help out, we can always use your help to improve the troop and the Scouting experience. This is YOUR troop. Help us, Help you, by Delivering the Promise of Scouting to OUR Scouts.



Yours in Scouting,\\ 
Richard C Wink\\ 
Committee Chairman\\ 
1 March 2023

\newpage
\tableofcontents

\newpage

\pagenumbering{arabic}

\section{TROOP 1853 OVERVIEW AND ORGANIZATION}

\subsection{TROOP 1853 HISTORY AND OVERVIEW}
Boys Troop 1853 was chartered in 1969 and Girls Troop 1853 was chartered in 2019. Our goal is to help achieve the mission of Scouting: “To prepare young people to make ethical and moral choices over their lifetimes by instilling in them the values of the Scout Oath and Law.” This includes providing all Scouts the opportunity to earn the rank of Eagle Scout through the implementation of the Aims and Methods of Scouting. Together, \ac{BT} and \ac{GT} have directly affected the lives of several hundred, if not several thousands of youth living in the West Springfield area. We strive hard to have a Scout-led troop.

\subsubsection{TROOP 1853 WEEKLY MEETINGS}
Troop 1853 meets weekly on Tuesday nights from 7:30 PM to 9:00 PM at the Community Covenant Church, unless otherwise announced in advance (e.g., Swimming pool night, special program locations, summer camp week, etc). The troop calendars have meeting locations and times. Exceptions to this are discussed in section \ref{TROOP MEETINGS}.

\subsubsection{TROOP 1853 MONTHLY CAMPOUTS AND SUMMER CAMP}
The troop outdoor program includes a monthly weekend campout (Friday night to around noon Sunday). The Scouts select these monthly activities during the annual planning conference. Each summer, usually in July, the troop attends a one-week summer camp. This camp is especially important for new Scouts and provides an excellent opportunity to work on basic rank requirements and merit badges; it also provides opportunities for older Scouts to try new merit badges or activities.

\subsection{DISTRICT AND COUNCIL}
Troop 1853 is part of the \ac{ODD} and the \ac{NCAC}. We participate in camporees with the \ac{ODD} and occasionally with other Scouting organizations, such as the West Point Camporee, other Districts, and \ac{NCAC}. This is planned by the Scouts during the annual planning conference.

\subsection{TROOP 1853 CHARTERED ORGANIZATION AND REPRESENTATIVE}
The \ac{BSA} does not operate Scout troops. The \ac{BSA} charters organizations to use the program as a resource for youth and families. Because the program of the \ac{BSA} is conducted only through chartered organizations, it is imperative that these  receive effective and meaningful support. Our Chartered Organization is the Community Covenant Church at 7018 Sydenstricker Road, Springfield, VA 22152. The Chartered Organization Representative is selected by the Chartered Organization. Contact with the Chartered Organization Representative is normally through the Troop Committee Chair. The Community Covenant Church Pastor(s) are the Institutional Head and they appoint the Chartered Organization Representative. In some circumstances the Church Committee Chair can/will serve as the institutional Head. The Chartered Organization Representative Guidebook (511-42116) from the \ac{BSA} is a further resource on the roles of the Chartered Organization.

\subsection{TROOP 1853 ORGANIZATIONAL STRUCTURE}
The \ac{BT} and \ac{GT} follow the standard \ac{BSA} Troop organizational structure. This may be modified as practical to meet the needs of the organization by the Troop Committee.

\subsubsection{TROOP RESPONSIBILITY, POSITIONS, FUNCTIONS AND PROCESSES}
\paragraph{TROOP COMMITTEE}
\subparagraph{RESPONSIBILITY}
The Committee’s primary responsibility is to support both the \ac{BT} and \ac{GT} leadership in delivering a quality program. These responsibilities include setting policy, helping the troop leadership with the outdoor program and other planned activities, staffing key functions such as advancement and registration, and ensuring financial support. If needed, the Committee acts as arbiter between the \ac{GT} and \ac{BT} Scoutmasters. The committee is also responsible for convening \acp{BOR}  for the advancement of Scouts. More information on \acp{BOR} is found in section 7.5. %ref

\subparagraph{POSITIONS}
The following are key committee positions the troop desires filled at all times to allow proper support to the troop’s Scouting program. The Troop 1853 Scouter Position Guide lists additional committee positions, their job descriptions and essential job functions.

\begin{itemize}
	\item Chartered Organization Representative - The Chartered Organization Representative is the direct contact between the committee and the Chartered Organization. The Chartered Organization appoints the unit committee chair and Scoutmasters.

	\item Troop Committee Chair - The troop committee chair is appointed by the chartered organization to see that all committee functions are carried out. The troop committee chair appoints and supervises the unit committee and unit leaders, and organizes the committee to see that all committee responsibilities are delegated, coordinated and completed. There is only one Troop Committee Chair.

	\item Troop Treasurer - The Troop Treasurer is appointed by the committee chair to handle troop funds, pay bills, maintain accounts, and create the yearly budget. The Troop Treasurer will oversee the escrow program, escrow coordinator and troop fundraising activities. The Troop Committee Chair may appoint a \ac{BT} and \ac{GT} Troop Treasurer.

	\item Troop Advancement Chair - The unit advancement chair is appointed by the committee chair to ensure both the \ac{BT} and \ac{GT} have at least monthly \acp{BOR}, periodic courts of honor, and have goals of helping each Scout advance in rank. They support the troop goal by informing the Scoutmaster of new Scouts to reach First Class rank during their first year. The advancement chair is also responsible for record keeping and submitting advancement reports. The Troop Committee Chair may appoint a \ac{BT} and \ac{GT} Troop Advancement Chair.

	\item Troop Membership Chair - The troop membership chair is appointed by the committee chair to ensure that the unit is chartered annually and to ensure that new youth and adult applications along with funds are completed and turned in to the committee chair for processing. The membership chair will support the troop’s recharter, recruitment, and retention functions. The Troop Committee Chair may appoint a \ac{BT} and \ac{GT} Troop Membership Chair.

	\item Troop \ac{BOR} Chair- Serves as the focal point for Scouts to request \acp{BOR}. Other tasks include : Collects the required documentation; publishes dates for \acp{BOR}; solicits \ac{BOR} members; briefs members on their duties; ensures acp{BOR} are held according to \ac{BSA} rules and according to Troop 1853 standards; provide feedback on advancement progress and \ac{BOR} comments from the youth to the committee/\ac{SM}.

	\item Scoutmaster – The Scoutmaster of the \ac{BT} and \ac{GT} is nominated by the committee to the Charter Organization who will make the formal appointment. As a member of the committee, the Scoutmasters are responsible for briefing the activities, advancement, service, and other aspects of Scouting for the respective \ac{BT} and \ac{GT}.

	\item Charter Organization – The charter organization, in our case, the Community Covenant Church, annually agrees to the following responsibilities:
	\begin{enumerate}
		\item Appoint the Troop Committee Chair, \ac{BT} and \ac{GT} Scoutmasters, and First and Second \acp{ASM}.
		\item Provides a meeting place and promotes a good Scouting program.
		\item Conduct the Scouting program consistent with \ac{BSA} rules, regulations, and policies.
		\item Approve adult membership applications.
	\end{enumerate}
\end{itemize}

\subparagraph{FUNCTIONS}
The Troop Committee has several positions and a number of functional areas. At times some of these positions may be filled by \acp{ASM}, but the goal is to create active committee members who supports both the \ac{BT} and \ac{GT} at the committee level. These functions include:

\begin{itemize}
	\item Ensures quality adult leadership. Additionally, ensures that all adult leadership is approved, registered, trained and current in Youth Protection Training and position specific training.

	\item Selects and nominates the Scoutmaster, the First \ac{ASM}, and Second \ac{ASM} to the Charter Organization. In case those positions are unfilled, absent for an extended period of time, or an individual is unable to serve, a qualified \ac{ASM} is appointed until the committee can recruit a replacement.

	\item Ensures troop activities are following the \ac{BSA} Guide to Safe Scouting (\url{https://www.scouting.org/health-and-safety/gss/}) and Youth Protection guidelines (\url{https://www.scouting.org/health-and-safety/gss/gss01/}).

	\item Advises the Scoutmasters on policies relating to Scouting and the chartered organization.

	\item Is responsible for finances to include ensuring adequate funds and disbursements are in line with the approved budget plan.

	\item Obtains, maintains, and properly cares for troop property.

	\item Ensures the troop has a quality program that includes an outdoor program (minimum ten days and nights per year). This is reviewed and approved annually.

	\item Ensures members serve on \acp{BOR} and support courts of honor.

	\item Supports the Scoutmasters in working with individual youth and problems that may affect the overall troop program.

	\item Provides for the special needs and assistance some youth may require.

	\item Assists the Scoutmasters with handling youth behavioral problems.
\end{itemize}
The Troop Handbook is updated and approved at a minimum of every five years or sooner at the discretion of the Committee. The Troop 1853 Handbook is approved by the Committee.

\subparagraph{PROCESS}
The committee meets once a month, usually the first Monday of the month. Any Scout or Scouter registered with the \ac{BT} or \ac{GT} may attend. Any parent or guardian with a Scout registered in the \ac{BT} or \ac{GT} is encouraged to attend. Voting members of the committee are those adults registered with either troop in a committee member status. A Scouter fulfilling a Troop 1853 committee responsibility but registered as an \ac{ASM} or another position within the troop are welcome and encouraged to attend but are not voting members. All \ac{BT} and \ac{GT} meetings are open to any parent or guardian of a registered \ac{BT} or \ac{GT} Scout and any registered Scouters of \ac{BT} or \ac{GT}.

\subparagraph{UNPUBLISHED AD HOC POLICIES}
From time to time, both the Scout and adult leaders of the \ac{BT} or \ac{GT} may establish temporary policies that are not reflected in this handbook. It is the responsibility of the policy sponsor to promptly bring these policies to the attention of the Troop Committee for its  ratification, amendment, or termination.

\paragraph{SCOUTMASTER AND \ac{ASM} POSITIONS,  RESPONSIBILITIES, AND FUNCTIONS}
\subparagraph{SCOUTMASTER}
The Scoutmaster provides leadership to the youth through the Patrol Leader's Council (\ac{PLC}) and to the \acp{ASM} by providing guidance and activity coordination. The \ac{SM} serves as the bridge between the \ac{PLC} and the Troop Committee.

\ac{BT} and \ac{GT} has a "First Assistant Scoutmaster" and a “Second Assistant Scoutmaster.” The "First Assistant Scoutmaster," acts in the absence of the Scoutmaster. The “Second Assistant Scoutmaster” acts in the absence of the Scoutmaster and the First Assistant Scoutmaster. Normally the First Assistant Scoutmaster and the Second Assistant Scoutmaster “move-up” when the Scoutmaster’s term is over every two years. The new Scoutmaster then selects a new "Second Assistant Scoutmaster" who must be nominated by the committee to the Charter Organization, who makes the final appointment.

\subparagraph{ASSISTANT SCOUTMASTERS}
The Scoutmaster designates \acp{ASM} who serve in the advisor role for select positions. These could include, but are not limited to the Patrol Advisor, Venture Program Coordinator, Troop Instructor Advisor, and the Order of the Arrow Advisor. Additional \acp{ASM} may be designated to assist with functions such as New Scout Training, Introduction to Leadership Skills for Troops (ILST), and Religious Awards.

An \ac{ASM}, Scouter, Parent or Guardian is assigned to each outdoor event and is responsible for the planning      and implementation of the event based on objectives established by the \ac{PLC}. This \ac{ASM}, Scouter, or Parent is designated the \ac{SMIC} and is assisted by other adults and the \ac{SPL} and or a senior Scout designated by the Scoutmaster. The Annual Planning Conference and \ac{PLC}s will assist the \ac{SMIC} and \ac{SPL} in refining the outdoor event. The \ac{BT} and \ac{GT} website has resources to assist the \ac{SMIC} in the execution of their duties.

The \acp{SM} and the \acp{ASM} have the responsibility of working with the youth as they advance along the Trail to Eagle. All Scouters, Parents and Guardians, and Scouts have responsibility in facilitating execution of the troop program.

\Acp{ASM} may be assigned specific duties by the Scoutmasters to assist the Troop Committee.

\Acp{ASM}, being the Adults with the most direct contact with the Scouts, are expected to obtain and wear a complete Scout Uniform and set the standard for uniform wear.

\subparagraph{ASSISTANT SCOUTMASTERS - PATROL ADVISORS}
\begin{itemize}
	\item The Scoutmaster will assign one or more \acp{ASM} to each patrol as Patrol Advisors.
	\item They advise the Patrol Leader on the operation of the patrol; they do not lead the patrol.
	\item Patrol Advisors ensure that training on Scout skills, leadership and outdoorsmanship is conducted properly, and guide the Patrol Leader with training and advancement as required. The Patrol Advisor works through the Patrol Leader on Patrol activities, advancement, and responsibilities (troop program, opening and closing ceremonies, outings, courts of honor, etc.).
	\item The Patrol Advisors guide the employment of Troop Guides in support of the Patrol Leader and patrol activities, provide periodic input on their performance, and recommend to the Scoutmaster whether or not they should receive leadership credit for their term.
	\item Teaching, coaching, mentoring, and evaluation of Scout Spirit are critical functions of a Patrol Advisor. The Scout Spirit rank requirement, from Scout to First Class rank, should be signed off by the Patrol Advisor prior to the Scout completing the Scoutmaster Conference.
\end{itemize}

\subparagraph{ASSISTANT SCOUTMASTERS - POSITION ADVISORS}
\begin{itemize}
	\item \acp{ASM} are assigned by the Scoutmaster to each of the troop officers who performs a specialized function (Order of the Arrow Representative, Scribe, Quartermaster, Librarian, Historian, Bugler, Outdoor Ethics Guide, and Chaplain Aide). The Scoutmaster will assign the \ac{ASM} to be a Position Advisor. Note: An \ac{ASM} may have more than one advisory role.
	\item These advisors are responsible to assist Scouts with their assigned Scout position as listed in the Troop 1853 Scout Leadership, Positions of Responsibility, and Election Process Guide.
	\item Position advisors monitor the performance of the Scout, provide coaching and instruction as required, and periodically counsel the Scout on their progress.
	\item At the conclusion of the Scout’s term, the position advisor recommends to the Scoutmaster whether or not the Scout successfully met the standards of their job.
\end{itemize}

\subparagraph{EAGLE COORDINATOR / MENTOR / COACH}
On reaching the Life rank, each Scout in either troop is assigned a Troop 1853 Eagle Coach to assist them in navigating the steps to the rank of Eagle Scout. The Eagle Coach is a resource and mentor through the Eagle process. The National Capital Area Council provides an Eagle Scout Procedures Guide, which can be found at: \url{https://www.ncacbsa.org/eagle-scout-information/}.

\subparagraph{PATROL LEADERS' COUNCIL} 
The \ac{PLC}, not the adult leaders, is responsible for planning and conducting the troop's activities. The Scoutmasters with support from the \acp{ASM} provide direction, coaching, and training that empower the \ac{PLC} members with the skills they will need to lead the troop. The Troop Committee provides resources to help the \ac{PLC}.

The \ac{PLC} constitutes the Scout leadership for Troop 1853. There are separate \acp{PLC} for both the \ac{BT} and \ac{GT}. The \ac{SPL}, \acp{ASPL}, the Patrol Leaders, Troop Guides, Troop Instructors, and the Troop Officers (Quartermaster; Chaplain Aide; Historian; Librarian; Bugler, \ac{TOAR}, Outdoor Ethics Guide, and Scribe) comprise the \ac{PLC}. See Section 1.4 (ADD REFERENCE) Troop Organizational Structure for a graphical and hierarchical depiction. While all Scouts may attend the \ac{PLC}, those listed above are required to attend or be excused by the Scoutmaster.

The Troop 1853 Scout Leadership, Position of Responsibility, and Troop Election
Process Guide (available on the websites) provides detailed information on Troop Leadership Position Information and Eligibility Requirements to include roles and responsibilities. It highlights the training and attendance requirements for each leadership position. The Troop 1853 Election process is explained in the Troop Election Process Guide. The Troop 1853 Scout Leadership, Position of Responsibility, and Troop Election Process Guide is on the troop website under references.

The \ac{PLC} plans the yearly troop program at the annual planning conference,  usually held in June. The \ac{PLC} meets monthly to develop plans for upcoming meetings and activities. All troop leadership is expected to attend both the annual planning conference and the monthly \ac{PLC} meetings in order to receive leadership credit.

\subparagraph{GREEN BAR}
Green Bar History. The Green Bar is named for William “Green Bar Bill” Hillcourt. Mr. Hillcourt wrote many of the best resources for Scouts and Scouters starting with handbooks for Scoutmasters and Patrol Leaders in the 1920’s. Green Bar Leadership Positions historically are based upon the Scout Leadership patches containing a “green bar” and are the \ac{SPL}, Troop Guides, Patrol Leaders, and \ac{APL}s.

In Troop 1853, we consider the \ac{SPL}, \acp{ASPL} and in certain situations, the Senior Scout a Green Bar. The \ac{JASM} is included when the Scoutmaster, in coordination with the Troop Committee, appoints a \ac{JASM}. The \ac{SPL} and \acp{ASPL} are not assigned to a Patrol as they are considered troop level leadership.

On campouts and other troop events, Troop 1853 Green Bar may eat and share camp responsibilities with the adult’s Patrol. Troop Guides, Patrol Leaders and \ac{APL}s will eat and camp with their assigned patrol.

\subparagraph{PIRATES PATROL}
The activation and deactivation of the Pirates Patrol or other ‘senior Scout’ patrol is at the discretion of the Scoutmaster in consultation with the Committee Chairman.
Historically a \ac{BT} patrol, the Pirates Patrol provides Scouts in the troop who have served in specific leadership positions within Troop 1853 an opportunity to build and operate their own program. To belong to the Pirates Patrol is a privilege, not a right. The Pirates Patrol member must meet eligibility listed below and receive approval of the Scoutmaster. The Scoutmaster may add or remove a Scout assigned to the Pirates Patrol. 

The Pirates Patrol program, while remaining tied into the overall troop program, allows the patrol members to participate in activities and programs that are more challenging and meaningful to their needs and interests. The Pirates Patrol will support the patrol and troop programs and lend their experience with the intent to improve overall program success.

\textit{Standards}
\begin{itemize}
	\item As a minimum, the Pirates Patrol will adhere to the same standards and rules as the regular patrols and will actively participate, as defined by the SM, in the troop program. They may be assigned a specific program month by the PLC.
	\item Troop 1853 expects that due to their previous leadership experience and rank they will exceed these standards and serve as role models for all other Scouts in the troop.
	\item On selected occasions, special privileges may be afforded to the Pirates as determined by the \ac{SPL}, \ac{SM}, or \ac{SMIC}, as appropriate, in keeping with the special trust afforded based upon their previous service and leadership to the troop.
\end{itemize}

\textit{Eligibility}
\begin{itemize}
	\item Scouts will be eligible to join the Pirates Patrol if they:
	\begin{itemize}
		\item have successfully served as a patrol leader, \textit{and} as either SPL or ASPL
		\item are not currently serving as PL of another patrol, SPL or ASPL
		\item are at least Life Rank
		\item receive approval of the \ac{SM}
	\end{itemize}
	\item The Scoutmaster may allow other Scouts not meeting the requirements above to join the patrol when there are special circumstances.
	\item Members of the Pirates Patrol who desire to serve as Troop Guide, Instructor, or Den Chief will be encouraged to do so. They will be expected to fulfill all of their responsibilities in these positions.
\end{itemize}

\textit{Program}
\begin{itemize}
	\item The Pirates Patrol’s program will be built on service and support to Troop 1853.
	\item The Pirates Patrol may have periodic patrol activities.
\end{itemize}

\textit{Camping:} 
The Pirates Patrol will have their own patrol equipment and will participate in campouts as their own patrol, cooking their own meals and setting up their own campsite. Pirates may tent individually and use hammocks from April to October. At the discretion of the SMIC, based upon campout program considerations and in coordination with the Scoutmaster they may merge with the adult’s Patrol for the campout.

\textit{Participation:}
The Pirates Patrol is a troop-level asset that the SPL, ASPLs, Patrol Leaders, Patrol Advisors, SMIC, the Scoutmaster and others within the troop leverage to assist in the planning and execution of other programs, activities and advancements. Pirate Patrol members may fill in as a de facto Troop Instructor.

Pirates Patrol members will direct and assist with troop activities. These examples include:
\begin{itemize}
	\item Assisting the PLC to lead ILST.
	\item Assisting with new and newly assigned Scouts.
	\item Overseeing Tenderfoot physical fitness tests.
	\item Assisting with training for Fireman Chit and Tot’n Chip.
	\item Assisting with summer camp preparation.
	\item Teaching Scout Skills on campouts, summer camp, and other troop events.
\end{itemize}

The Pirates Patrol will not participate in the Troop Honor Patrol competition. However, they are expected to set the example for the other Patrols and to assist their assigned Patrol if they are a Troop Guide. They may participate in earning the recognition of a National Honor Patrol.

\subsubsection{ADDING/REMOVING PATROLS}
The size of the troops continually grows and shrinks. At the discretion of the SM, in consultation with the Committee Chairman, they will adjust the number of patrols to accommodate this fluctuation. Optimally, patrol size should be 8-10 active Scouts. This allows for a sufficient number of Scouts for the patrol leader to guide, without placing an excessive burden on them. This size for patrols also allows for adequate opportunities for leadership positions. The \ac{SM} will determine who meets the “active” criteria. Generally, it includes frequent participation in meetings, activities and campouts.

\section{TROOP GUIDEPOSTS}
\subsection{GUIDING PRINCIPLES}
The guiding principles of Troop 1853 are the \ac{BSA} Mission, Aims and Methods of Scouting. The Troop 1853 goal is to maximize opportunities for Scouts to experience the Scouting values while having fun.

\subsubsection{SAFETY}
Safety is a vital concern to the troop and all safety regulations and policies established by the \ac{BSA} and the Chartered Organization will be enforced. During all activities, Troop 1853 will follow \ac{BSA}’s Guide to Safe Scouting, which can be found at:  \url{https://www.scouting.org/health-and-safety/gss/}  Scouts who repeatedly violate safety rules may be sent home from the activity. Safety is everyone’s responsibility.

\subsection{MISSION}
The mission of the \ac{BSA} is to prepare young people to make ethical choices over their lifetime by instilling in them the values of the Scout Oath and Law.

\subsection{AIMS}
%The Scouting program has specific objectives, commonly referred to as the “Aims of Scouting.” They are \textbf{character development, leadership development, citizenship training, and personal fitness}. Leadership development is also one of Scouting’s eight methods contributing to both good character and good citizenship.

\subsection{METHODS}
The methods by which the Aims are achieved are listed below in random order to emphasize the equal importance of each.

\subsubsection{IDEALS}
The ideals of Scouting are spelled out in the Scout Oath, the Scout Law, the Scout motto, and the Scout slogan. The Scout measures themselves against these ideals and continually tries to improve. The goals are high, and, as they reach for them, they have some control over what and who they become.

\subsubsection{PATROLS}
The patrol method gives Scouts an experience in group living and participating citizenship. It places responsibility on young shoulders and teaches Scouts how to accept it. The patrol method allows Scouts to interact in small groups where they can easily relate to each other. These small groups determine troop activities through their elected representatives.

\subsubsection{OUTDOOR PROGRAMS}
Scouting is designed to take place outdoors. It is in the outdoor setting that Scouts share responsibilities and learn to live with one another. It is here that the skills and activities practiced at troop meetings come alive with purpose. Being close to nature helps Scouts gain an appreciation for God.

\subsubsection{ADVANCEMENT}
Scouting provides a series of surmountable obstacles and steps in overcoming them through the advancement method. The Scout plans their advancement and progresses at their own pace as they meet each challenge. The Scout is rewarded for each achievement, which helps them gain self-confidence. The steps in the advancement system help a Scout grow in self-reliance and in the ability to help others.

\subsubsection{ASSOCIATION WITH ADULTS}
Scouts learn a great deal by watching how adults conduct themselves. Scout leaders can be positive role models for the members of their troops. In many cases a Scoutmaster who is willing to listen to the Scouts, encourage them, and take a sincere interest in them can make a profound difference in their lives.

\subsubsection{PERSONAL GROWTH}
As Scouts plan their activities and progress toward their goals, they experience personal growth. The Good Turn concept is a major part of the personal growth method of Scouting. Young people grow as they participate in community service projects and do Good Turns for others. Probably no device is so successful in developing a basis for personal growth as the daily Good Turn. The religious emblems program also is a large part of the personal growth method. Frequent personal conferences with their Scoutmaster help each Scout to determine their growth toward Scouting.

\subsubsection{LEADERSHIP DEVELOPMENT}
The Scouting program encourages Scouts to learn and practice leadership skills. Every Scout has the opportunity to participate in both shared and total leadership situations. Understanding the concepts of leadership and becoming a servant leader helps a Scout accept the leadership role of others and guides them towards participating citizenship and character development.

\subsubsection{UNIFORM}
The uniform makes the Scout troop visible as a force for good and creates a positive youth image in the community. Scouting is an action program, and wearing the uniform is an action that shows each Scout for Scout activities and provides a way for Scouts to wear the badges that show what they have accomplished. Scouts and \acp{ASM} are expected to make every effort possible to wear the correct uniform for each function. Leadership will make every effort to publish and inform everyone of any changes. 

\paragraph{THE TROOP 1853 - \ac{BSA} UNIFORM}
Normally any Scouting event associated with Troop 1853 will require either the Field Uniform or the Activity Uniform. A good rule of thumb is to wear the Activity Uniform shirt under the Field Uniform shirt, that way the Scout has both uniforms just in case. The troop will make accommodations to help Scouts obtain uniform pieces if they need assistance and have a demonstrated need. We have a uniform locker in the basement in which to donate or purchase uniforms items. Each uniform item is \$5, paid to the Troop Treasurer or Troop Uniform Coordinator. The T-Shirt coordinator can also order you additional Troop 1853 activity shirts (long or short sleeve), fleeces, and hats. Troop 1853 believes that a full uniform is an important part of the program and expects Scouts to obtain and wear one.

\subparagraph{FIELD UNIFORM (unofficially known as the “Class A” Uniform)}
Scouts wear the field uniform to troop meetings and traveling to/from all Scouting events, including campouts. The Activity Uniform may be worn to troop meetings during the summer months, but this will be announced ahead of time. The Field Uniform is mandatory all year long for the Scoutmaster Conference, Board of Review, and Courts of Honor. The full Field Uniform is listed below. In Troop 1853 Scout lingo, ‘being full’ means you are wearing all of the uniform pieces. After the start of each troop meeting, the Patrol Leader will report the number of patrol members present and if they are in full uniform. If it comes down to attending the meeting not in uniform or not attending the meeting at all, we want your Scout to attend the meeting.

Field Uniform
\begin{itemize}
	\item \ac{BSA} official long- or short-sleeve tan Shirt with green shoulder loops on epaulets
	\item \ac{BSA} Troop 1853 Neckerchief (supplied by the troop), other Scout-related neckerchief, or Scout related Bolo Tie
	\item \ac{BSA} official olive-colored Pants or Shorts; no cuffs or olive Skort for the \ac{GT} (at the discretion of the SMs, olive colored Pants or Shorts may be substituted for the official \ac{BSA} items)
	\item \ac{BSA} official Socks or hiking socks if wearing hiking boots
	\item Official Scout web Belt with \ac{BSA} insignia on buckle, or official leather belt with buckle of your choice or \ac{BSA} official pants or Shorts with sewn in belt.
	\item Hiking Boots or other appropriate footwear (i.e. close-toed shoes). Hiking boots are mandatory on campouts.
\end{itemize}

\subparagraph{ACTIVITY UNIFORM (unofficially known as the “Class B” Uniform)}
This is worn for daily routines during Scout functions, during campouts, Scouting activities, summer time, and any “work” type of Scout activity. The activity uniform is a fill-in uniform for when the field uniform is not appropriate. The Patrol Leader or Senior Patrol Leader will decide if these are required for a particular meeting or event, or if any other type of shirt or item will be permitted.

Activity Uniform
\begin{itemize}
	\item \ac{BSA} Troop 1853 T-shirt or other Scout related T-shirt
	\item \ac{BSA} official olive-colored Pants or Shorts; no cuffs, or olive Skort for the \ac{GT} (at the discretion of the SMs, olive colored Pants or Shorts may be substituted for the official \ac{BSA} items)
	\item \ac{BSA} official Socks or hiking socks if wearing hiking boots
	\item Official Scout web Belt with \ac{BSA} insignia on buckle, or official leather belt with buckle of your choice or \ac{BSA} official pants or Shorts with sewn in belt.
	\item Hiking Boots or other appropriate footwear (i.e. close-toed shoes)
\end{itemize}

\subparagraph{OTHER}
\begin{itemize}
	\item Sandals, flip-flops, or open-toed shoes MAY NOT be worn with any uniform or on campouts. Some activity areas (swimming pools, bath houses, etc.) may require an exception to this rule; such an exception will be announced in advanced by the Scoutmaster (Note- the expectation is that Adults participating or attending these functions set a proper example as well)
	\item Hats and jackets are acceptable and encouraged for outdoor activities. Hats are generally NOT worn indoors – especially at the church. The weather will usually determine the best type of hat to wear (i.e. brimmed or stocking cap). Scout-inappropriate logos will not be worn.
\end{itemize}

\subparagraph{\ac{BSA} UNIFORM SHEET}
The \ac{BSA} Uniform inspection sheet can be found at on the troop websites. This provides further details on patch  locations and other uniform items. It can also be found in the Scout Handbook.

\subsection{SCOUT SPIRIT}
What is Scout Spirit? - The way in which a Scout demonstrates their commitment to Scouting is through living by the Scout Oath and Law. Doing a daily Good Turn is one of the best ways in which a Scout can show their Scout Spirit. Scouts also demonstrate Scout Spirit when they participate in service projects of the troop, when they help the troop with troop and patrol events and activities, and by providing service to their school, their respective churches, mosques, synagogues, and their community as a whole.

\subsection{STANDARD OF CONDUCT}
All Scouts, Scouters, Merit Badge Counselors, parents, and friends are expected to live by the Scout Oath and Law. This must be followed both in daily conduct, and in the resolution of any disagreements. Behavior and discipline are discussed further in Chapter 10 Troop 1853 Rules of Behavior and Discipline Policy / Procedures (ADD REFERENCE).

\subsection{DISPUTE RESOLUTION}
In every organization, there are disputes; they are inevitable and should be dealt with as quickly as possible.

The first alternative is to talk over the dispute with the individual(s) involved.
Scouts should try to resolve their own problems rather than the parents -- its part of the training program. If the Scouts cannot resolve the problem themselves, they should then go to their Patrol Leader, the Senior Patrol Leader, and then the Scoutmaster (or Patrol Advisor) if the situation is not resolved.

Parents should discuss their concerns with the patrol advisors and if not resolved, should discuss them with the Scoutmaster. Scouters should talk first to the individuals involved, then the Scoutmaster. If the dispute is not resolved, then the parent should present the issue to the Troop Committee for resolution.

\subsection{ACTIVITIES AND ITEMS NOT ALLOWED}
The troop does not permit Scouts to use sheath knives or folding knives with blades over 3 1/2 inches long. Scouts must earn the Tot'n Chip to use and carry a knife. Scouts must exercise extreme care in properly storing their folding knife at home. For example, Scouts should use care when using their day pack for both school and Scout activities so they don’t accidently bring a knife to school. Scouts may only bring two knives, including a multi tool type tool, to weekend and camp activities.

Use of entertainment devices are only allowed in vehicles when traveling to and from a troop activity. They may not be brought into camp even if they are not used there. Scouts will not have firearms or any items which are otherwise illegal such as fireworks, vaping  devices, cigarettes, or drug paraphernalia. Inappropriate reading or pornographic material is  also not allowed.

Under no circumstances is anything which contains a flame allowed in tents, shelters, or other prepared structures.

Scouts are forbidden from departing the activity area without the express permission of one of the adult leaders. Sign-out sheets will be maintained by the SMIC to account for those Scouts leaving and returning from other activities (sports games and school events for example). Departures must be coordinated in advance and the adult picking up the scout should be authorized to do so by a legal guardian of the Scout.

\subsection{TROOP 1853 TECHNOLOGY POLICY}
\subsubsection{GENERAL GUIDELINES}

\begin{itemize}
	\item Use the technology to build relationships with the troop, find useful information, communicate, and share excitement about Scouting.
	\item Updates to social sites using appropriate, (non-embarrassing), photos or clips can share and build excitement about Scouting.
	\item Don’t let technology detract from the outdoor experience, the program experience, or the Scouting experience for the troop or patrol.
	\item Teach Technology use. We don’t ban acceptable knives, hand axes or gas lanterns; we teach their use. Similarly, we don’t ban technology. We teach its use.
\end{itemize}

\subsubsection{GENERAL USE}
Troop 1853 supports the use of technology as a useful tool. The Scout Law should guide the use of technology devices. The device should not be a distraction. Scouts should pay attention to the program and fellow Scouts. In a program or troop situation, a Scout should avoid checking their device for incoming messages or postings. Scouts should not become overly reliant on the device. For example, a Scout should be ready with their map and compass rather than rely on a device’s GPS.

\subsubsection{TECHNOLOGY RESPONSIBILITY}
The care, use, security, cost, and power requirements of the technology device is the responsibility of the Scout. The Scout is responsible for any and all costs associated with the use of the technology device; these may include voice, text, data plans, and application purchases or use. The Scout may request to secure their technology device in a vehicle or lock box during the campout. The Troop will assume no responsibility or liability for the loss or damage of any  device. When in doubt, leave them at home.

\subsubsection{TECHNOLOGY DISCIPLINE AND REPERCUSSIONS}
Should any \ac{PL}, \ac{SPL}, \ac{ASM} / \ac{SM}, determine the technology in use is not in support of the activity or being properly used in accordance with the Scout Law they will request the Scout put away the device until the end of the activity. The \ac{SPL}, \ac{SM}, or \ac{SMIC} may determine that the technology device should be taken away from the Scout for the remainder of the event. If taken away the device will be secured in a vehicle or lock box. If the use of a device continues to be problematic, the \ac{SM} will discuss the issue with the Scout’s parent.

\section{TROOP 1853 MEMBERSHIP, RECHARTER, RECRUITMENT AND RETENTION}
\subsection{TROOP RECHARTERING, DUES, REGISTRATION}
Rechartering is conducted by the Troop Membership chair on an annual basis. All troop members will support the annual process by completing training, paying dues, and updating information as requested by the membership chair. Any Scout or Scouter who expects to be registered with Troop 1853 during the annual rechartering must complete Youth Protection Training and pay their dues. Any Scouter who is not current in their Youth Protection Training may not reregister. All Scouters should work to become fully trained in their role. 

\subsubsection{ANNUAL DUES AND FEES}
Dues and Fees are reviewed and set each year by the Unit Committee. This is normally done during the September Committee meeting. The fees portion pays for the annual registration fee to the National Council of \ac{BSA} and \ac{BSA} Insurance. A subscription for Scout Life magazine may be included for an additional fee and is encouraged as a great source of Scouting information. The dues portion covers troop operating/administrative expenses and capital replacement costs. It does not pay for campouts, summer camp, or high adventure activities. Those are self-sustaining. 

Registration fees for Scouters are also established by the unit committee. These include all \ac{BSA} fees and insurance and include a dues portion. Those registering with the \ac{ODD} as Merit Badge Counselors, Nova Counselors and Supernova mentors are not charged an annual registration fee.

\subsection{TROOP RECRUITING}
Troop 1853 actively recruits from Cub Scout Packs in the local area. Historical success has been achieved through an active Webelos Liaison and Recruiter who reaches out to Cubmasters and Webelos Den Leaders early in the Scouting/School year during the Webelos Arrow of Light Year during September through November timeframe. This allows Den Leader to plan for and visit the troop on a timeline that supports their crossover schedule.

Troop 1853 will usually offer two Webelos Days during troop meeting times. They are focused on Webelos visiting the troop and the meeting is designed to showcase the troop, its Scouts, how we conduct Scouting, and to help the Webelos and their parents determine if the troop is a good fit. The Webelos Liaison and Recruiter coordinate with the \ac{SM} and the \ac{PLC} to inform  them of when Arrow of Light Scouts are visiting the troop so the Program can be planned accordingly. The Webelos days are normally planned for October and February. Additionally, Troop 1853 welcomes any potential new or transferring Scouts to visit during any troop meeting.

Webelos Crossover. Den Leaders and or Cubmaster should advise the Webelos Liaison and Recruiter and or the \ac{SM} of the date of their crossover ceremonies. The \ac{SM}, First \ac{ASM}, or the Second \ac{ASM} will attend the crossover ceremony along with the \ac{SPL} or \ac{ASPL} and the \ac{PL} of the new Scout’s Patrol to welcome the Webelos crossing over to Troop 1853. Crossing over Webelos will receive a unit neckerchief, slide/woggle. They will also receive information to complete the transfer to the troop.

Once a Scout has registered and paid their dues, they will receive the remainder of their Scout package. This includes a Scout Handbook, current scout epaulets, patrol patch, Troop 1853 Patch and any other items determined by the troop.

\subsection{SCOUT RETENTION}
The Scoutmaster will review the participation of all Scouts periodically. Scouts not participating over the past several months will be identified and the Patrol Advisors will attempt to determine the Scout’s intentions. The Scoutmaster will report to the Troop Committee any systemic issues that are identified which may require corrective action by the troop, the first action then being a special Board of Review as outlined in section 8.0.1.3 of the 2017 \ac{BSA} Guide to Advancement.

\section{TROOP 1853 COMMUNICATIONS}
\subsection{OVERVIEW}
The Troop has several means of communications that include E-mail, Phone calls, announcements at troop meetings, the troop website, and the quarterly “The Eagle” newsletter.

Face to Face communication remains the best form of communication especially between the Scout and Scouter. However, the Troop uses email as it primary communication method, and as such, the Troop does not encourage the use of texting, especially when it is between a Scout and an adult Scout leader. Scouts should refrain from texting adult scout leaders and adults scout leaders should refrain from texting Scouts. Should the need arise for a Scout or Scouter to text, they should include another adult on the text to ensure they follow the \ac{BSA} no one-on-one contact rules. 

The troop has established an email that should be included on all adult to scout communications to ensure no one-on-one discussions. This is to protect both the Scout and the Adult. 

Boys Troop: ypt@troop1853.org
Girls Troop: TBD

This address is delivered to several Adult Scouters who monitor these communications according to the principles of the Guide to Safe Scouting and Youth Protection Guidelines. This also includes other phone or computer applications in which a Scouter or adult could contact the Scout directly without knowledge by the parent or other adult. Time sensitive communication should be handled by a phone call.

\subsection{EMAIL}
Information about activities (e.g., event times, dates, locations, uniform, etc.) comes out in emails from Scouts or Scouters in charge of the activity. These emails should be able to answer most questions about an activity. If you are NOT receiving information via email or you  change your email address, please send an email to:

Boys troop: email@troop1853.org
Girls Troop: TBD

Include all the email addresses you want to be added or removed, and they will update the appropriate distribution list. Any Scout’s email address will be added to the troop and the Scout’s respective Patrol distribution lists with parent or guardian permission.

Scouts will need an email address or regular access to email. The Scout will need to communicate with their Patrol, \ac{SM}, and other adults or Scouts on certain activities and events. Families will need to plan for access. Will the Scout have their own personal email account or will they share a parent’s email? The Scout / Parent should regularly check the email for troop and Patrol information. Email is a critical information dissemination method used by the troop. Be prepared for a lot of emails – the troop errs on the side of over communicating. When the Scout is sending emails to adults they should include a parent or can use the Youth Protection email in the troop Contacts section to ensure there is no one-on-one contact

E-mail groups are used for general announcements (e.g., upcoming meetings, campouts, events). The addresses for these groups will not be provided on the troop website and should not be placed in individual address books. This helps maintain the security of the list and in doing so reduces SPAM.

\subsection{PHONE CALLS}
A phone call should be used for time-sensitive announcements or when you need to solve a problem immediately. Phone numbers are listed in the troop directory. When leaving a voicemail message, include your name, patrol, and phone number. The availability and frequency of Directory updates depends on an adult volunteer being available to generate the Directory. 

\subsection{ANNOUNCEMENTS DURING TROOP MEETINGS}
In addition to email, information is also put out at troop meetings at the start and end of meetings. If you are picking up or dropping off your Scout, you should plan to stay for one or both of these information / announcement sessions. Usually, the person in charge of an event is the one making the announcement at the troop meeting. This is a great time to ask questions. If you are making an announcement it should be less than one minute, additional information can be distributed to the Troop via email or flyers.

\subsection{THE TROOP NEWSLETTER}
“The Eagle” is published every quarter to keep Scouts, Scouters and parents informed and up to date on troop activities.

\begin{itemize}
	\item The newsletter is normally emailed to the troop. It is also posted on the troop web site.
	\item Each issue contains a calendar of events for the next few months. Also included are the Patrol duties for the coming month, news of upcoming activities, reports of recent activities, the Scoutmaster's Minute, and a report from the Troop Committee Chairman. The Eagle editor will solicit articles a few weeks prior to the publication date. 
	\item Many leadership positions require the Scout to publish one or more articles in The Eagle.
\end{itemize}

\subsection{TROOP CALENDAR}
The troop calendars are available:

Boys Troop: \url{https://www.troop1853.org/event/}

Girls Troop: \url{https://troopwebhost.org/Troop1853SPRINGFIELD/}

You can subscribe to the calendar so troop events show up in your calendar app. 

\subsection{TROOP 1853 WEBPAGE}
Boys Troop: \url{https://www.troop1853.org/}

Girls Troop: \url{https://troopwebhost.org/Troop1853SPRINGFIELD/}

are the troop websites maintained by the Troop Committee to provide an always- available source of information about the troops. In addition, copies of forms (Honor Patrol Worksheets, \ac{PLC} Patrol Forms, permission slips and medical exam forms) are usually available for downloading. Only the first names of Scouts will be shown on the troop  websites.

\subsection{SOCIAL MEDIA}
All Social Media within the troop will follow BSA Policy Social Media Policy. 
\url{https://scoutingwire.org/marketing-and-membership-hub/social-media/social-media- guidelines/}

\subsubsection{TROOP 1853 FACEBOOK PAGE}
The troop committee is responsible for the Troop 1853 Facebook page. This page will be administered by Troop 1853 registered Scouters in accordance with the BSA Social Media Policy and any Chartered Organization policies. Facebook page members and invitees must be registered with the troop or in some way linked to Troop 1853. These may include parents of current Scouts, former Scouts and or their parents, former Scouters, and or district or council Scouters related to Troop 1853.

\subsubsection{TROOP ALUMNI FACEBOOK PAGE}
The troop committee is responsible for the Troop 1853 Alumni Facebook page. This page focuses mainly on past troop activities in an attempt to keep former members of the troop connected with the troop, and serve as a visual repository of the troop’s history. All former members are encouraged to post photographs and comment on those posted. The page features an Eagle Scout Roll Call album, and a collection of past Eagle newsletters.

\subsection{TROOP DIRECTORY}
A troop directory may be published in the spring and fall each year after Troop Elections to reflect new leadership positions. The Directory is only published when an adult volunteer is available and willing to put it together. It should include the address, phone number and e-mail address of the Scouts and the adult leaders, and the rank and age of each Scout. Scouts are listed both alphabetically and by patrol. If no Troop Directory is published it is inherent on Troop Scout Leadership to ensure individual patrol information is shared. Scoutbook may also be used to obtain Scout, Scouter, and adult information.

\subsection{TROOP 1853 INDIVIDUAL AND GROUP CONTACTS}
Should you need to contact Troop 1853 the best person is the Scoutmaster at 
scoutmaster@Troop1853.org 
or the Committee Chair at 
committee\_chair@Troop1853.org.

Other persons of note for the \ac{BT} are listed below:

\begin{tabular}{ l l }
	\textbf{Person or Group} & \textbf{Address} \\
	\hline
	Activities Recorder 				& activities\_recorder@troop1853.org \\
	Advancement Chair 				& advancement\_chair@troop1853.org \\
	Everone (Scouts, Parents, Leaders)	& announce@troop1853.org \\
	Assistant Scoutmasters			& asms@troop1853.org \\
	Board of Review Chair			& bor@troop1853.org \\
	Troop Committee				& committee@troop1853.org \\
	Committee Chair				& committee\_chair@troop1853.org \\
	Troop Zelle Account				& deposits@troop1853.org \\
	Dragons Patrol					& dragons@troop1853.org \\
	Eagle Editor					& eagle\_editor@troop1853.org \\
	Email Administrator				& email@troop1853.org \\
	Greenbar Patrol				& greenbar@troop1853.org \\
	New Scout Recruiting			& info@troop1853.org \\
	All registered adult leaders		& leaders@troop1853.org \\
	Medical Form Coordinator			& medical\_forms@troop1853.org \\
	Membership Chair				& membership\_chair@troop1853.org \\
	Active OA Members				& oa@troop1853.org \\
	Scorpion Patrol					& scorpions@troop1853.org \\
	Scoutmaster					& scoutmaster@troop1853.org \\
	Shark Patrol					& sharks@troop1853.org \\
	Adult Training Chair				& training\_chair@troop1853.org \\
	Treasurer						& treasurer@troop1853.org \\
	Unit College Reserve Members		& unitcollegereserves@troop1853.org \\
	Webelos Liaison				& webelos\_liaison@troop1853.org \\
	Website Administrator			& webmaster@troop1853.org \\
	Wolf Patrol					& wolves@troop1853.org \\
	Youth Protection Email Monitors	& ypt@troop1853.org \\
	\hline
\end{tabular}

Other persons of note for the \ac{GT} are listed below:

\begin{tabular}{ l l }
	\hline
	\textbf{Person or Group} & \textbf{Address} \\
	All Adults 						& all\_adults.Troop1853SPRINGFIELD@twh.email \\
	All Scouts 					& scouts.Troop1853SPRINGFIELD@twh.email \\
	All Troop 						& all\_troop.Troop1853SPRINGFIELD@twh.email \\
	Assistant Scoutmasters			& ASMs.Troop1853SPRINGFIELD@twh.email \\
	CCC Leadership 				& cccleadership.Troop1853SPRINGFIELD@twh.email \\
	Committee 					& committee.Troop1853SPRINGFIELD@twh.email \\
	Committee Chair 				& committee.chair.Troop1853SPRINGFIELD@twh.email \\
	Flaming Cacti 					& cacti.Troop1853SPRINGFIELD@twh.email \\
	Info 							& info.Troop1853SPRINGFIELD@twh.email \\
	Night Owls 					& owls.Troop1853SPRINGFIELD@twh.email \\
	Sales 						& sales.Troop1853SPRINGFIELD@twh.email \\
	Scouters 						& scouters.Troop1853SPRINGFIELD@twh.email \\
	Scoutmaster 					& scoutmaster.Troop1853SPRINGFIELD@twh.email \\
	Scouts Only 					& scouts.only.Troop1853SPRINGFIELD@twh.email \\
	Site Admin 					& siteadmin.Troop1853SPRINGFIELD@twh.email \\
	Treasurer 						& treasurer.Troop1853SPRINGFIELD@twh.email \\
	\hline
\end{tabular}


\section{TROOP 1853 MEETINGS AND CEREMONIES}
\label{TROOP MEETINGS}


\section{TROOP 1853 TRAINING}


\section{TROOP 1853 ADVANCEMENT (LEADERSHIP, RANK AND MERIT BADGES)}


\section{TROOP 1853 PROGRAM AND PROGRAM SUPPORT}


\section{TROOP 1853 FINANCES}


\section{TROOP 1853 RULES OF BEHAVIOR AND DISCIPLINE POLICY PROCEDURES}



\section{Acronyms}
\begin{acronym}[XXXXXXX]
	\acro{APL}{Assistant Patrol Leader}
	\acro{ASM}{Assistant Scoutmaster}
	\acro{ASPL}{Assistant Senior Patrol Leader}
	\acro{BOR}{Board of Review}
	\acro{BSA}{Boy Scouts of America}
	\acro{BT}{Boys Troop 1853}
	\acro{COR}{Chartered Organization Representative}
	\acro{GT}{Girls Troop 1853}
	\acro{NCAC}{National Capital Area Council}
	\acro{NHP}{National Honor Patrol}
	\acro{OA}{Order of the Arrow}
	\acro{ODD}{Old Dominion District}
	\acro{OTC}{Over the Counter}
	\acro{PL}{Patrol Leader}
	\acro{PLC}{Patrol Leaders Council}
	\acro{SM}{Scoutmaster}
	\acro{SMIC}{Scoutmaster in Charge}
	\acro{SPL}{Senior Patrol Leader}
	\acro{STEM}{Science, Technology, Engineering, and Mathematics}
	\acro{TOAR}{Troop Order of the Arrow}
\end{acronym}

\end{document}